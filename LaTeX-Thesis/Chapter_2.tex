\chapter{Literature Review} \label{chap:literature}

In this chapter it will be referenced previous literature and related research to this thesis.
It's worth noticing that neural networks have been developed as a medium of defining
problem solving in a defined fashion similar to the structure of the human brain.

Neural networks have emerged as a ubiquitous model in machine learning, due their flexibility 
to adapt to very different types of problems, and handling a numerous amount of variables.
From prediction models to pattern recognition, image processing, or generative models, NN and DL are
techniques that are defined to learn from data, allowing them to do data-driven decisions.
Learning from experience, machines therefore can solve problems without the definition
of a predefined model, but in doing so, at least with a classical implementation, such as
CNN, the definition of the model is then encoded within the a set of neurons, called hidden layer,
makes it more difficult to easily interpret or change the model without re-training the 
neural network with other data that could adapt the output of the model.

For the setup of the neural networks and tools to be used in the implementation
the following components were used:

\subsection{Keras}  \label{subSec:Keras}

Keras is a framework for easy and fast protyping of neural networks 
that can run in CPU or GPU's, excelent for deep learning. It runs using
Python and works as an abstraction layer for another backend suchs as
Tensorflow and Theano. 

Keras contains numerous functions for building blocks such as layers, 
objective functions, activation functions, optimizers, and a set of tools to facilitate 
the work with image and text.

