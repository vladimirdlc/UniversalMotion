\chapter{Literature Review} \label{chap:literature}

In this chapter it will be referenced previous literature and related research to this thesis.
It's worth noticing that neural networks have been developed as a medium of defining
problem solving in a defined fashion similar to the structure of the human brain.

Neural networks have emerged as a ubiquitous model in machine learning, due their flexibility to adapt to very different types of problems, and handling a numerous amount of variables.
From prediction models to pattern recognition, image processing, or generative models, NN and DL are techniques that are defined to learn from data, allowing them to do data-driven decisions.
Learning from experience, machines therefore can solve problems without the definition
of a predefined model, but in doing so, at least with a classical implementation, such as
CNN, the definition of the model is then encoded within the a set of neurons, called hidden layer, makes it more difficult to easily interpret or change the model without re-training the neural network with other data that could adapt the output of the model.

For the setup of the neural networks and tools to be used in the implementation
the following components were used:

\subsection{Keras}  \label{subSec:Keras}

\citet{Keras:2018} is a framework for easy and fast protyping of neural networks 
that can run in CPU or GPU's, excelent for deep learning. It runs using
Python and works as an abstraction layer for another backend suchs as
Tensorflow and Theano. 

Keras contains numerous functions for building blocks such as layers, 
objective functions, activation functions, optimizers, and a set of tools to facilitate 
the work with image and text.

\subsection{Tensorflow}  \label{subSec:Tensorflow}

As defined in their github repository \citet{Tensorflow:2018} "TensorFlow is an open source software library for numerical computation using data flow graphs.
The graph nodes represent mathematical operations, while the graph edges represent the multidimensional data arrays 
(tensors) that flow between them. This flexible architecture enables you to deploy computation to one or more CPUs or GPUs in a desktop, 
server, or mobile device without rewriting code. TensorFlow also includes TensorBoard, a data visualization toolkit."

\subsection{Autoencoders}  \label{subSec:Autoecoders}

One of the several applications of neural networks is the definition of an architecture defined as Autoencoders. Autoencoders are structures that are compromise of at least an enconding layer, a number of hidden layers and a decoder, where the hidden space is decoded as output.

This type of algorithms have several applications from denoising, to reconstruction, and  definition of manifolds depending of the input of the raw data provided. 

\subsection{Neural Networks}  \label{subSec:NeuralNetworks}



\subsection{Convolutional Networks}  \label{subSec:ConvolutionalNetworks}

As \citet{rashid2016make} NN's emerged from a drive for biologically inspired computers, similar to human brains.  
Specifically \citet{Goodfellow-et-al-2016-Book} mentions that Convolutional Neural Networks (CNN's), is the more classic configuration for neural networks, that is specialized 
for processing data that has a known grid-like topology. Examples can be time-series data, prediction series, and image data. 
The name “convolutional neuralnetwork” indicates that the network employs a mathematical operation called convolution, that consist of a linear operation 
over the nodes with a rectifier function on the output of each operation that works as an input to another neuron.


\subsection{BVH}  \label{subSec:BVH}

The Biovision Hierarchy (BVH) character animation file format was developed by Biovision,   
that have a format with the addition of a hierarchical data structure 
representing the bones of the skeleton. The BVH file consists of two parts
where the first section details the hierarchy and initial pose of the skeleton and the second section
describes the channel data for each frame, thus the motion section.
It's a widely accepted format that due the simplicity of its structure can be used in several
3d animation software.

\subsection{Formats of Representation of Animation} \label{subsec:AnimationRepresentationFormats}

Animation can be stored in different representation formats, each with different characteristics such as lenght of representation,  format or complexity of the data stored.

Some of represented items are as follow:

\begin{itemize} 
	\item  Axis Angle.
	This representation system is compromised of 4 values, 3 representing a vector and another additional one representing 
	the angle $\theta$ that defines the number of the degrees to rotate such vector.
	\item  Rotation Matrices
	
	We can consider the following set of matrices, for a given euler angles yaw, pitch and roll, each matrix
	represent a different rotation over a different axis.
	
	Rz($\alpha$) = 
	$\begin{pmatrix} 
	cos \alpha & -sin \alpha & 0 \\ 
	sin \alpha & cos \alpha & 0 \\ 
	0 & 0 & 1  
	\end{pmatrix} $
		
	Ry($\beta$) = 
	$\begin{pmatrix} 
	cos \beta & 0 & sin \beta \\ 
	0 & 1 & 0 \\ 
	-sin \beta & 0 & cos \beta  
	\end{pmatrix} $
	
	Rx($\gamma$) = 
	$\begin{pmatrix} 
	1 & 0 & 0 \\ 
	0 & cos \gamma & -sin \gamma \\ 
	0 & sin \gamma & cos \gamma  
	\end{pmatrix} $ 
	\bigbreak \bigbreak 
	Therefore the 3x3 combination of
	R($\alpha$,$\beta$,$\gamma$) = Rz($\alpha$)*Ry($\beta$)*Rz($\gamma$)
	
	represents a rotation under a 3d space.
	
	\item  Quaternions
	Quaternions is a very stable number system that extends complex numbers.
	
	\item  Euler Angles
	
\end{itemize}
