\chapter{Introduction} \label{chap:introduction}
 Chapter \ref{chap:introduction} provides a brief summary of the elements to be used in this thesis. A comprehensive literature review on the topic of Neural Networks and Autoencoders is presented in Chapter \ref{chap:literature},
 showcasing from introductory concepts to more advanced techniques and their applications.

 Deep Learning is an area of application of the computer science that has been growing more and more throughout the years.
 The possibilities of applications, the adaptability and the support for different types of problems is something that quite remarkably 
 had not been able to be achieved with other types of algorithms or techniques in the area of Machine Learning.
 
 In comparison with other techniques such a Reinforcement Learning (RL), Neural Networks and more specifically deep neural networks,
 are a source of new problem solving that was impossible to achieve before, due the intrincany of works and number of variables 
 of the models that could be worked with other learning techniques. The adpatability of DL to solve from problems of classification, to some others of such as prediction is what 
 have developed an interest in this kind technology with passing of years. 
 
 According to \citet{Zeng:2008} a standard neural network (NN) consists of many simple, connected processors called neurons, each producing a sequence of real-valued activations. 
 Input neurons get activated through sensors perceiving the environment, other neurons get activated through weighted connections from previously active neurons. 
 Some neurons may influence the environment by triggering actions. Learning or credit assignment is about finding weights that make the NN exhibit desired behavior, 
 such as driving a car. Depending on the problem and how the neurons are connected, such behavior may require long causal chains of computational stages, 
 where each stage transforms (often in a non-linear way) the aggregate activation of the network.
  
 \citet{Holden:2014} worked in a project with similar motion data definition, he specified that motion is typically represented as a time-series where each frame 
 represents some pose of a character. Poses of a character are usually parametrized by the character joint angles, or joint positions. 
 Also, he believe that this representation is excellent for data processing, and that valid human motion only exists in a small subspace of this representation. 
 
 In this research is provided an analysis of a series of storing techniques for animations that after being parsed by convulutional neural networks will provide a reconstruction
 of its input. The main format utilized in this research to store the animation is BVH, due the simplicity of use, and the database defined to use this is the CMU.
 Further details of the implementation and the structure will be explained in Chapter 3, as well as the indicators of comparison between all of them, using as main indicators
 the possible loss and the MSE validation error between the original and the reconstructions.
 
 In this research is provided an analysis of a series of storing techniques for animations that after being parsed by convulutional neural networks will provide a reconstruction
 of its input. The main format utilized in this research to store the animation is BVH, due the simplicity of use, and the database defined to use this is the CMU.
 Further details of the implementation and the structure will be explained in Chapter 3, as well as the indicators of comparison between all of them, using as main indicators
 the possible loss and the MSE validation error between the original and the reconstructions.
 
 
 
 \section{Figure and Table} \label{sec:FigureAndTable}

 Text body of the Section \ref{sec:FigureAndTable}.
 
 \subsection{Figure}  \label{subSec:Figure}

 A figure example is shown in Figure \ref{fig:Compare}.
 
 %%%%%%%%%%%%%%%% begin figure %%%%%%%%%%%%%%%%%%%
 \begin{figure}
     \begin{center}
         \includegraphics[scale=0.6]{Compare}  %it is suggested to using scale to zoom figures without distortion
        \end{center}
        \caption{An illustration of requirement compliance.}
        \label{fig:Compare}
    \end{figure}
%%%%%%%%%%%%%%%% end figure %%%%%%%%%%%%%%%%%%%
    

\subsection{Table}  \label{subSec:Table}
    
 Table \ref{table:ROM_elements} illustrates a very complex table with figures in its cells.
 
 \begin{table}[htp]
     \small{
         \caption{Elements defined for the ROM \citep{Zeng:2008}.}
         \begin{center}
             \label{table:ROM_elements}
             \begin{tabular}{p{1.3cm} p{1.8cm} p{2.1cm} p{4.5cm}} \hline \hline
                 \multicolumn {2}{c}{Type} &  Graphic Representation  & \multicolumn {1}{c}{Description} \\\hline
                 \multirow {3}*{Object} & Object & \hfil \raisebox{-0.35cm}{\includegraphics[scale=.5]{FigInTable1_1}} \hfil & Everything in the universe is an object \\
                 & Compound Object & \hfil \raisebox{-0.5cm}{\includegraphics[scale=.5]{FigInTable1_2}} \hfil & It is an object that includes at least two objects in it\\\hline
                 \multirow {7}*{Relation} &  Constraint Relation & \hfil \raisebox{-0.6cm}{\includegraphics[scale=.4]{FigInTable1_3}} \hfil & It is a descriptive, limiting, or particularizing relation of one object to another\\
                 & Connection & \hfil \raisebox{-0.45cm}{\includegraphics[scale=.4]{FigInTable1_4}} \hfil & It is to connect two objects that do not constrain each other \\
                 & Predicate Relation & \hfil \raisebox{-0.6cm}{\includegraphics[scale=.4]{FigInTable1_5}} \hfil & It describes an act of an object on another or that describes the states of an object \\\hline \hline
                \end{tabular}
            \end{center}
        }
    \end{table}
    
    
\section{Itemized examples using list structures in \LaTeX{}}
\label{SubSec.:bullet}

Item list using ``\texttt{itemize}" structure are given below:

% itemize style 
\begin{itemize} 
    
    \item  Use bold/italic for emphasis, but keep its use to a minimum. Avoid using underlining in your paper.
    \item  Use a consistent spelling style throughout the paper (US or UK).
    \begin{itemize}  % level 2 items
        \item  Use double quotes.
        \item  Use \%, not percent.
        \item  Do not use ampersands (\&) except as part of the official name of an organization or company.
    \end{itemize}
    \item  Keep hyphenation to a minimum. Do not hyphenate `coordinate' or `non' words, such as `nonlinear'.
    
\end{itemize}

The following are using ``\texttt{enumerate}" structure:

% enumerated style 
\begin{enumerate} 
    
    \item  For complete or near complete sentences, begin with a capital letter and end with a full stop.
    \item  For short phrases, start with lower case letters and end with semicolons.
    
\end{enumerate}

    
\section{Algorithm}
    
    The pseudo code shown in Algorithm \ref{alg:myAlgo} describes the proposed algorithm.
    
    \begin{algorithm}
        \small{
            \caption{Calculate the probability of $G$}\label{alg:myAlgo}
            \begin{algorithmic} [1]
                \Require $p \in [0,1]$, $G$
                \Ensure None
                \For{$i = 0 \to 2^d-1$}\Comment{$d$ is an integer} 
                \If{$n(\nu_i) = 0$}
                \If{ $x < p$}  \Comment{$x$ is a normal distribution number in the range of $[0,1]$}
                \State Occupy $v_i$ site with probability $p$ 
                \EndIf
                \EndIf
                \EndFor
            \end{algorithmic}}
        \end{algorithm}
        
        
  \section{Equation}
        \label{section:equation}
        
        An equation example is shown in Eq. \ref{equ:enc}.
        
        \begin{equation}\label{equ:enc}
        f(ENC)=\int_0^{1}(e^{x}+x^{2})
        \end{equation}
        
\section{Quotations}
        \label{subSec:quotation}
        
        \begin{quote}
            ``It was easier in the beginning when there was only the RED-camera, but now, after RED, it just continuous. And all the different manufacturers, they cannot agree upon what is the standard file format, codec, or compression algorithms, and so on. It is a jungle."
        \end{quote}
        \begin{flushright} CEO, Full Name (Company A) \end{flushright}
        

\section{Citations}
        \label{subSec:Citations}
        
        It is suggested that you choose ``\textbf{$\backslash$citet}" and/or ``\textbf{$\backslash$citep}" to cite references. The ``\textbf{$\backslash$citet\{key\}}" gives you a format of  ``\textbf{Name (1990)}", whileas ``\textbf{$\backslash$citep\{key\}}" delivers a format of ``\textbf{(Name, 1990)}". For example, \citet{Wang&Zeng:2009} extended their research from \citep{Zeng:2008}.
        